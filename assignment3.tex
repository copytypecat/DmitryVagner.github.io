\documentclass{amsart}

%\usepackage{etex}
%\usepackage{graphicx}
%\usepackage{amsfonts}
%\usepackage[usenames,dvipsnames]{xcolor}
%\usepackage[bookmarks=true,colorlinks=true, linkcolor=RoyalBlue, citecolor=magenta]{hyperref}
\usepackage{subfig}
\usepackage{mathtools}
\usepackage{amssymb}
\usepackage{dsfont}
\usepackage{cmll}
\usepackage{url}
\usepackage{bbm}
\usepackage{stmaryrd}
\usepackage{tikz}
\usepackage{tikz-cd}
\usepackage{bussproofs}
\usepackage{enumerate}

%Theorem Environments%%%%%%%%%%%%%%%%%%%%%%%%%%%%%%%%%%%%%%%%%%%%%%%%%%%%%%%%%%%%%%%%%%%%%%%%%%%%%%%

\newtheorem{thm}{Theorem}[section]
\newtheorem{lem}[thm]{Lemma}
\newtheorem{cor}[thm]{Corollary}
\newtheorem{prop}[thm]{Proposition}

\theoremstyle{remark}
\newtheorem{rem}[thm]{Remark}

\theoremstyle{definition}
\newtheorem{dfn}[thm]{Definition}
\newtheorem*{notation*}{Notation}

\theoremstyle{definition}
\newtheorem{ex}[thm]{Example}

%Definitions%%%%%%%%%%%%%%%%%%%%%%%%%%%%%%%%%%%%%%%%%%%%%%%%%%%%%%%%%%%%%%%%%%%%%%%%%%%%%%%%%%%%%%%%
\DeclareMathOperator{\aut}{Aut}
\DeclareMathOperator{\List}{List}
\DeclareMathOperator{\irr}{Irr}
\DeclareMathOperator{\lcm}{lcm}
\DeclareMathOperator{\Char}{char}
\DeclareMathOperator{\gal}{Gal}
\DeclareMathOperator{\tr}{Tr}
\DeclareMathOperator{\id}{id}
\DeclareMathOperator{\sgn}{sgn}
\DeclareMathOperator{\colim}{colim}
\DeclareMathOperator{\Ker}{Ker}
\DeclareMathOperator{\im}{Im}
\DeclareMathOperator{\Hom}{Hom}
\DeclareMathOperator{\spa}{span}
\DeclareMathOperator{\Ext}{Ext}
\DeclareMathOperator{\Sym}{Sym}
\DeclareMathOperator{\Res}{Res}
\DeclareMathOperator{\Ind}{Ind}
\DeclareMathOperator{\ad}{ad}
\DeclareMathOperator{\Ob}{Ob}
\DeclareMathOperator{\mat}{Mat}


\def\taking{\colon}
\def\ot{\leftarrow}
\def\too{\longrightarrow}
\def\inj{\hookrightarrow}
\newcommand{\an}[1]{\langle #1 \rangle}
\newcommand{\ceil}[1]{\lceil #1 \rceil}
\newcommand{\floor}[1]{\lfloor #1 \rfloor}
\def\rla{\rightleftharpoons}
\newcommand{\To}[1]{\xrightarrow{#1}}
\newcommand{\From}[1]{\xleftarrow{#1}}
\def\ss{\subseteq}
\newcommand{\ol}[1]{\overline{#1}}
\newcommand{\Id}[1]{\mathbbm{1}_{#1}}
\def\then{\sslash}

\def\op{^{\text{op}}}
\def\Set{\mathbf{Set}}
\def\Rel{\mathbf{Rel}}
\def\Fin{\mathbf{Fin}}
\def\Vect{\mathbf{Vect}}
\def\Pos{\mathbf{Pos}}
\def\Pre{\mathbf{Pre}}
\def\Tot{\mathbf{Tot}}
\def\Mon{\mathbf{Mon}}
\def\Com{\mathbf{Com}}
\def\Grp{\mathbf{Grp}}
\def\Cat{\mathbf{Cat}}
\def\mcC{\mathcal{C}}
\def\mcD{\mathcal{D}}
\def\mcG{\mathcal{G}}
\def\mcL{\mathcal{L}}
\def\bfL{\mathbf{Lin}}
\def\bfW{\mathbf{W}}
\newcommand{\Opd}[1]{\mathcal{O}#1}

\def\BP{\mathbf P}
\def\BQ{\mathbf Q}
\def\PP{\mathbb P}
\def\BB{\mathbb B}
\def\NN{\mathbb N}
\def\ZZ{\mathbb Z}
\def\QQ{\mathbb Q}
\def\RR{\mathbb R}


\newcommand{\pic}[2]{\includegraphics[height=#1]{#2}}

\setcounter{section}{2}

\begin{document}

\section{Assignment 3}

\subsection{Problem 1}

Let $\mcC$ be a category with terminal object $\star$. Define $\widetilde{\mathcal{C}}$ to have objects $(f,x_0)\in\mathcal{C}(X,X)\times\mathcal{C}(\star,X)$ with arrows $(f,x_0)\to(g,y_0)$ given by $\mcC$-arrows $\varphi:X\to Y$ such that the following two diagrams commute.
\[
\begin{tikzcd}
X \arrow[r,"\varphi"] \arrow[d,"f"'] & Y \arrow[d,"g"] & & & \star \arrow[d,"x_0"'] \arrow[dr,"y_0"] & \\ X \arrow[r,"\varphi"'] & Y &&& X \arrow[r,"\varphi"'] & Y  
\end{tikzcd}
\]
\begin{enumerate}[(a)]
    \item Prove that $\widetilde{\mcC}$ is a category.
    
    \item Show that the objects of $\widetilde{\Set}$ can be seen as \emph{discrete dynamical systems}
    \[\begin{cases} x_{n+1} = f(x_n) \\
    x_0 \end{cases}\]
    
    \item Define the \emph{successor} map as $s:\NN\to\NN::n\mapsto n+1$. Prove that $(s,0)$ is the initial object in $\widetilde{\Set}$.
    
\end{enumerate}

\subsection{Problem 2}

Characterize the terminal object of $\mathcal{C}/c$ and initial object of $c/\mathcal{C}$.

\subsection{Problem 3}

Let $\mcC$ be a category with products such that $\mcC\op = \mcC$. 
\begin{enumerate}[(a)]
    \item Prove that $\mcC$ has biproducts; i.e. that it has coproducts and that these are isomorphic to products.
    
    \item Recall the category $\Rel$ consisting of objects sets and arrows $X\to Y$ subsets $R\subseteq X\times Y$, i.e. maps $X\times Y\to\BB$. Writing $xRy$ for $R(x,y)=\top$, we compose relations $R:X\to Y$ and $Q:Y\to Z$ by defining $x(QR)z$ if and only if there exists $y$ such that $xRy$ and $yQz$. Prove that the disjoint union $X+Y$ gives a product in this category.
    
    \item Prove that $\Rel = \Rel\op$. Conclude that $+$ gives a biproduct in $\Rel$. 
    
    \item This allows us to conceive of any relation as a matrix of relations between singletons $\{x\}R\{y\}$. What should the entries of this matrix look like?
    
    \item Interpret what matrix multiplication should mean in this context.
\end{enumerate}

\subsection{Problem 4}

Show that any equalizer is a monomorphism. Argue by duality that any coequalizer is an epimorphism.

\subsection{Problem 5}

Let $\mcC$ be a category with pullbacks and a terminal object $\star$. 
\begin{enumerate}[(a)]
    \item Let $X,Y$ be objects. Show that their product $X\times Y$ is isomorphic to the pullback of the following diagram.
    \[
    \begin{tikzcd}
     {} & X \arrow[d, dashed,"\exists!\varphi"]\\
     Y \arrow[r, dashed,"\exists!\psi"'] & \star 
    \end{tikzcd}
    \]
    
    \item Let $f,g:X\to Y$ and consider the following pullback diagram
    \[
    \begin{tikzcd}
     X\times_Y X \arrow[r,"\sigma_1"]\arrow[d,"\sigma_2"'] \arrow[dr, phantom, "\lrcorner", very near start]  & X \arrow[d,"f"] \\
     X \arrow[r,"g"'] & Y
    \end{tikzcd}
    \]
    Prove that the equalizer of $f,g$ is the pullback of the following diagram.
    \[
    \begin{tikzcd}
      {} & X \arrow[d,"{(\sigma_1,\sigma_2)}"] \\
      X \arrow[r,"{(\Id{X},\Id{X})}"'] & X\times X
    \end{tikzcd}
    \]
    Conclude that the existence of pullbacks and a terminal object implies the existence of products and equalizers. Now suppose $\mathcal{C}$ has products and equalizers.
    
    \item Consider the following diagram
    \[
    \begin{tikzcd}
     {} & X \arrow[d,"f"] \\
     Y \arrow[r,"g"'] & Z
    \end{tikzcd}
    \]
    Show that the pullback of this diagram is isomorphic to the equalizer of the following pair of parallel arrows.
    \[
    \begin{tikzcd}
    X\times Y\arrow[r,shift left = .5 ex,"\pi_X\then f"] \arrow[r, shift right = .5 ex, "\pi_Y\then g"'] & Z
    \end{tikzcd}
    \]
    \item Argue, as in assignment 1, why terminal objects should be nullary products.
    
    \item Conclude that a category $\mcC$ has pullbacks and a terminal object if and only if it has products and equalizers. Use duality to argue why $\mcC$ has pushouts and an initial object if and only if it has coproducts and coequalizers.
\end{enumerate}

\subsection{Problem 6}
This problem is a classic. Consider the following commutative rectangle, whose righthand square is a pullback. Prove that the left hand square is a pullback if and only if the whole rectangle is a pullback.

\[
\begin{tikzcd}
\bullet \arrow[r] \arrow[d] & \bullet  \arrow[dr, phantom, "\lrcorner", very near start]  \arrow[r] \arrow[d] & \bullet \arrow[d] \\
\bullet \arrow[r] & \bullet  \arrow[r] & \bullet
\end{tikzcd}
\]

\subsection{Problem 7}

Let $\ol{\NN}=\mathbf{T}(\NN,\leq)$. Consider the diagram $\mathcal{G}:\ol{\NN}\to\Set$ defined by $\mathcal{G}n=\mathbf{1}+X+X^2+\cdots+X^n$ and $\mathcal{G}[n\to n+1]$ the canonical inclusion map $\mathbf{1}+X+X^2+\cdots+X^n\to \mathbf{1}+X+X^2+\cdots+X^n+X^{n+1}$. We show $\colim\mathcal{G}=\List X$.
\begin{enumerate}[(a)]
    \item Show that the maps \[\iota_n:\mathbf{1}+X+\cdots+X^n\to\List X::(x_1,\dots,x_n)\mapsto[x_1,\dots,x_n],\]
    where $\iota_0:\mathbf{1}\to\List X::0\mapsto [\:]$, define a cocone to $\List X$. In other words, check the necessary commutativity requirements.
    \item Now suppose there is some other cocone with components $\varphi_n:\mathbf{1}+\cdots+X^n\to Z$. Define a map $\Phi:\List X\to Z$ for which $\iota_n\then\Phi = \varphi_n$. 
\end{enumerate}


\subsection{Problem 8}

Let $(P,\preceq)$ be a preorder whose associated thin category is denoted by $\ol{P}$. Let $x\in P$ and consider the hom functor $\ol{P}(x,-):P\to\Set$. This takes values either $\star$ or $\varnothing$ and hence can be reconceptualized as a monotone map:
\[\ol{P}(x,-):P\to\BB.\]
\begin{enumerate}[(a)]
    \item Show that $\mathcal{U}_x=\ol{P}(x,-)^*(\top)$.
    \item In this context, the Yoneda Lemma states:
    \[\Pre(\ol{P}(x,-),\ol{P}(y,-))\cong\ol{P}(y,x)\]
    Use this, in conjunction with (a), to prove
    \[\mathcal{U}_x\subseteq\mathcal{U}_y \Rightarrow y\preceq x.\]
    \item Now suppose $\ol{P}(x,-)\cong\ol{P}(y,-)$. Show that this implies $x\cong y$.
\end{enumerate}


\end{document}