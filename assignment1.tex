\documentclass{amsart}

%Packages%%%%%%%%%%%%%%%%%%%%%%%%%%%%%%%%%%%%%%%%%%%%%%%%%%%%
\usepackage{subfig}
\usepackage{mathtools}
\usepackage{amssymb}
\usepackage{dsfont}
\usepackage{cmll}
\usepackage{url}
\usepackage{bbm}
\usepackage{stmaryrd}
\usepackage{tikz}
\usepackage{tikz-cd}
\usepackage{bussproofs}
\usepackage{enumerate}

%Definitions%%%%%%%%%%%%%%%%%%%%%%%%%%%%%%%%%%%%%%%%%%%%%%%%%%%%%%%%%%%%%%%%%%%%%%%%%%%%%%%%%%%%%%%%
\DeclareMathOperator{\lcm}{lcm}
\DeclareMathOperator{\Hom}{Hom}

\newcommand{\an}[1]{\langle #1 \rangle}
\newcommand{\ceil}[1]{\lceil #1 \rceil}
\newcommand{\floor}[1]{\lfloor #1 \rfloor}
\newcommand{\To}[1]{\xrightarrow{#1}}
\newcommand{\From}[1]{\xleftarrow{#1}}
\newcommand{\ol}[1]{\overline{#1}}
\newcommand{\Id}[1]{\mathbbm{1}_{#1}}

\def\Set{\mathbf{Set}}
\def\Pos{\mathbf{Pos}}
\def\op{^{\text{op}}}

\def\PP{\mathbb P}
\def\BB{\mathbb B}
\def\NN{\mathbb N}
\def\ZZ{\mathbb Z}
\def\QQ{\mathbb Q}
\def\RR{\mathbb R}

\def\ot{\leftarrow}
\def\too{\longrightarrow}
\def\inj{\hookrightarrow}
\def\rla{\rightleftharpoons}
\def\ot{\leftarrow}
\def\too{\longrightarrow}
\def\inj{\hookrightarrow}
\def\ss{\subseteq}
\def\then{\sslash}

\begin{document}

\section{Assignment 1}

\subsection{Problem 1}

In the context of a poset, use upper sets to prove that, when it exists, the join $\vee$ satisfies the following three properties.

\begin{enumerate}[(a)]
    \item $(x\vee y)\vee z = x\vee (y\vee z)$ \emph{(associativity)}
    \item $x\vee y = y\vee x$ \emph{(commutativity)}
    \item $x\vee x = x$ \emph{(idempotency)}
    \item Use duality to prove that the above also hold for $\wedge$. 
\end{enumerate}

\noindent These three properties respectively imply that bracketing, order, and multiple incidence don't affect joins and meets. Hence, for a lattice $(P,\preceq)$ and a \emph{finite} subset $S\subseteq P$, we can construct suprema and infima in terms of joins and meets:
\[\sup S = \bigvee_{s\in S}s \hspace{10 mm} \inf S = \bigwedge_{s\in S}s.\]
\begin{enumerate}[(a)]\setcounter{enumi}{4}
    \item Show that, in an infinite lattice, $\sup S$ and $\inf S$ need not exist for infinite $S$.
\end{enumerate}

\noindent Suppose there exist elements $\top$ and $\bot$ that satisfy the following.
\begin{align*}
    x\preceq \top &\text{ for all } x\in P \\
    \bot\preceq x &\text{ for all } x\in P
\end{align*}
We call $\top$ the \emph{top} element and $\bot$ the \emph{bottom} element.

\begin{enumerate}[(a)]\setcounter{enumi}{5}
    \item Prove that an equivalent definition for $\top$ and $\bot$ is that they satisfy \emph{unitality}:
   \begin{align*}
    x\vee \bot = x &\text{ for all } x\in P \\
    x\wedge \top = x &\text{ for all } x\in P
\end{align*}

    \item Use this to show that $\top,\bot$ can also be defined as suprema and infima:
    \[\top = \inf\varnothing \hspace{10 mm} \bot = \sup\varnothing\]
    \item Prove that, when $P$ is a finite lattice, $\top$ and $\bot$ can be computed as follows. \[\top = \sup P \hspace{10 mm} \bot = \inf P\]
    
    \item Show that, in an infinite lattice, $\top$ and $\bot$ need not exist.
\end{enumerate}

\subsection{Problem 2}

\noindent Let $\PP_n=p_0\cdot p_1\cdots p_{n-1}$, be the product of the first $n$ primes.
Construct an explicit isomorphism
\[(\mathcal{P}\mathbf{n},\subseteq)\To{\cong}(\an{\PP_n},|),\]
i.e. give rules for monotone maps $f,g$ of type
\begin{align*}
    f:\mathcal{P}\mathbf{n}&\to \an{\PP_n} \\
    g: \an{\PP_n} &\to \mathcal{P}\mathbf{n}
\end{align*}
and demonstrate that $g\circ f =\Id{\mathcal{P}\mathbf{n}}$ and $f\circ g=\Id{\an{\PP_n}}$.

\subsection{Problem 3}

\begin{enumerate}[(a)]
    \item Prove that
\[\operatorname{cof}:{\an{n}}\op\to \an{n}::a\mapsto\tfrac{n}{a}\]
forms a Galois connection with its opposite.

    \item Express the De Morgan laws for this Galois connection.
    
    \item Select $n$ in the above De Morgan laws so as to deduce the famous identity
    \[a\cdot b = \gcd(a,b)\cdot\lcm(a,b).\]
\end{enumerate}

\vspace{5 mm}

\subsection{Problem 4}
Let $f:X\to Y$ be a map and $f^*:\mathcal{P}Y\to\mathcal{P}X$ be its preimage.
\noindent Define the \emph{direct image} as:
    \[f_*:\mathcal{P}X\to\mathcal{P}Y::S\mapsto\{f(s)\mid s\in S\}\]
    
\begin{enumerate}[(a)]
    \item Show that $f_{*}$ is monotone.
    \item Show that $(f_*,f^{*})$ is a Galois connection.
    \item Write the two de Morgan laws implied by the above Galois connection.
    \item Provide a map $f:X\to Y$ for which there exist $A,B\subseteq X$ such that
    \[f_*(A\cap B)\neq f_*(A)\cap f_*(B).\]
    \item Prove directly that, in contrast
    \[f^*(A\cup B)=f^*(A)\cup f^*(B).\]
    \end{enumerate}
\noindent Lament this asymmetry. Now define the \emph{indirect image} as:
     \[f_{!}:\mathcal{P}X\to\mathcal{P}Y::S\mapsto \{y\in Y\mid f^*(y)\subseteq S\}.\]

\begin{enumerate}[(a)]\setcounter{enumi}{5}
    \item Show that $f_{!}$ is monotone.
    \item Show that $(f^*,f_{!})$ is a Galois connection.
    \item Write the corresponding de Morgan laws. What do you notice?
    \item Provide a map $f:X\to Y$ for which there exist $A,B\subseteq X$ such that
    \[f_{!}(A\cup B)\neq f_{!}(A)\cup f_{!} (B).\]
    \item Show that the direct image $f_*$ is equal to the map below
    \[f_{\exists}:\mathcal{P}X\to\mathcal{P}Y::S\mapsto \{y\in Y\mid \exists x\in f^*(y), x\in S\}.\]
    \item Show that the indirect image $f_!$ is equal to the map below
   \[f_{\forall}:\mathcal{P}X\to\mathcal{P}Y::S\mapsto \{y\in Y\mid \forall x\in f^*(y),x\in S\}.\]
    \item Discuss the higher symmetry that resolved what seemed to be an asymmetry.
 
\end{enumerate}

\end{document}